\documentclass[a4paper,12pt]{article}                                         % Schriftgröße, Layout, Papierformat, Art des Dokumentes
\usepackage[left=3cm,right=2cm,top=2cm,bottom=2cm,includehead]{geometry}      % Einstellungen der Seitenränder
%\usepackage{ngerman}                                                          % neue Rechtschreibung
\usepackage[ngerman]{babel}                                             % deutsche Silbentrennung
\usepackage[utf8]{inputenc}                                                   % Umlaute
\usepackage[hyperfootnotes=false, pdfborder={0 0 0}]{hyperref}                                   % pfd-Output [Fußnoten nicht verlinken]
\usepackage[nottoc]{tocbibind}                                                % Inhaltsverzeichniserweiterung (Inhaltsverzeichnis selbst ausblenden)
\usepackage{makeidx}                                                          % Index
%\usepackage[intoc]{nomencl}                                                   % Abkürzungsverzeichnis
\usepackage{fancyhdr}           	                                          % Fancy Header
\usepackage{pdfpages}															% For PDF include in the Appendix
\usepackage[round]{natbib}                                                    % Zitate (Erweiterung für Literaturverzeichnis)
\usepackage{amsmath}                                                          % Zurücksetzen der Tabellen- und Abbildungsnummerierung je Sektion
\usepackage[labelfont=bf,aboveskip=1mm]{caption}                              % Bild- und Tabellenunterschrift (fett)
\usepackage{setspace}                                                         % Zeilenabstand (vor footmisc laden!)
\usepackage[bottom,multiple,hang,marginal]{footmisc}                          % Fußnoten [Ausrichtung unten, Trennung durch Seperator bei mehreren Fußnoten]
\usepackage{wrapfig}
\usepackage{nomencl} % Für das Symbolverzeichnis
\usepackage{chngcntr} % Für die Nummerierung der Formeln in separaten Dateien
\usepackage{import} % Zum Einbinden von externen Dateien
% Umfließende Grafiken und Tabellen
\usepackage{graphicx}                                                         % Grafiken
\usepackage{tabularx}                                                         % erweiterte Tabellen
\usepackage{longtable}                                                        % mehrseitige Tabellen
\usepackage{color}                                                            % Farben
\usepackage{enumitem}                                                         % Befehl setlist (Zeilenabstand für itemize Umgebung auf 1 setzen)
\usepackage{listings}                                                         % Quelltexte
\usepackage{zref}                                                             % Verweise (Anhangsverweise)
\usepackage{wasysym}
\usepackage{amssymb}
\usepackage{xspace}
\widowpenalty=300
\clubpenalty=300
\usepackage{multicol}
\usepackage{booktabs}
\usepackage{siunitx}
\RequirePackage{multirow}
\RequirePackage{longtable}
\RequirePackage{mwe}
\RequirePackage{listings}    
\RequirePackage{etoolbox} 


\usepackage{float}
\usepackage{adjustbox}
\usepackage{multirow}
\usepackage{comment}
\usepackage{longtable}
\usepackage{caption}
\usepackage{subcaption}
\usepackage{graphicx}
\usepackage{hanging}
\usepackage{acronym}

%%%%%%%%%%%%%%%%%%%%%%%%%%%%%%%%%%%%%%%%%%%%%%%%%%%%%%%
%% Konfiguration %%
\def\myType{2}
%% 0=Projektarbeit   %%
%% 1=Bachelorarbeit  %%
%% 2=sonstige Arbeiten   %%
%% 3=sonstige Arbeiten mehrere Autoren  %%

\def\myTopic{Zusammenarbeitssysteme}
\def\mySubTopic{Handout}

\def\myStudyProgram{Data Science und künstliche Intelligenz}         % Studiengang
\def\myNeedForFieldOfStudy{} % Gibt an ob eine Studienrichtung benötigt wird
%% 0=keine Studienrichtung %%
%% 1=Studienrichtung %%
\def\myFieldOfStudy{}         % Studiengangsrichtung

\def\myAutor{Sinan, Valentin \& Finn}
\def\myCourse{Einführung in die Wirtschaftsinformatik}
\def\myProf{Dr.\ Marco van Baal}
\def\myCompany{}
\def\myCompanion{}            % Begleitperson des dualen Partners
\def\myEndDate{05.06.2025}              % Abgabefrist


%% Projektarbeit %%
\def\myProjNumber{1}          % [1|2]

%% Bachelorarbeit %%
\def\myTypeOfBachelor{}       % [Arts|Science]

%% sonstige Arbeiten %%
\def\myTypeOfWork{}           % [Seminararbeit|Projektdokumeantation|...]
\def\myModule{Einführung in die Wirtschaftsinformatik}               % Modulname
\def\myNames{1}               % Gibt an ob mehrere Personen an der Arbeit mitwirken
%% 0=Einzahl %%
%% 1=Mehrzahl %%

%% Konfiguration der Verzeichnisse %%
% für alle folgenden Konfigurationen gilt: 0=kein Verzeichnis | 1=Verzeichnis %

\def\myAbrevationDirectory{1}   % Abkürzungsverzeichnis
\def\myIllustrationDirectory{1} % Abbildungsverzeichnis
\def\myTableDirectory{1}        % Tabellenverzeichnis
\def\myListingDirectory{0}      % Listingverzeichnis
\def\myFormularDirectory{0}     % Formelverzeichnis
\def\mySymbolDirectory{0}      % Symbolverzeichnis
\def\myKiDirectory{0}           % KI-Verzeichnis

%% Konfiguration Selbständigkeitserklärung und Sperrvermerk %%
\def\myDecOfIndependence{1}
%% 0=keine Selbständigkeitserklärung   %%
%% 1=Selbständigkeitserklärung    %%

\def\myBlockingNote{0}
%% 0=kein Sperrvermerk   %%
%% 1=Sperrvermerk    %%

% Ort und Datum der Unterschrift auf Selbständigkeitserklärung und Sperrvermerk
\def\myDate{04.06.2025}
\def\myPlace{Ravensburg}

%%%%%%%%%%%%%%%%%%%%%%%% Eigene Farbwerte definieren %%%%%%%%%%%%%%%%%%%%%%%%
\definecolor{boxgray}{gray}{0.9}         % Hintergrundfarbe für Zitatboxen
\definecolor{commentgray}{gray}{0.5}     % Grau für Kommentare in Quelltexten
\definecolor{darkgreen}{rgb}{0,.5,0}     % Grün für Strings in Quelltexten

%%%%%%%%%%%%%%%%%%%%%%%% Eigene Kommandos definieren %%%%%%%%%%%%%%%%%%%%%%%%
% Definition von \gqq{#1: text}: Text in Anführungszeichen
\newcommand{\gqq}[1]{\glqq #1\grqq}

% Definition von \footref{#1: label}
% Verweis auf bereits existierende Fußnoten mittels
\providecommand*{\footref}[1]{
	\begingroup
		\unrestored@protected@xdef\@thefnmark{\ref{#1}}
	\endgroup
\@footnotemark}

% Definition von \mypageref{#1: label}
% Kombination aus \ref{#1: label} und \pageref{#1: label}
\newcommand{\mypageref}[1]{\ref{#1} \nameref{#1} auf Seite \pageref{#1}}

% Definition von \myboxquote{#1: text}
% grau hinterlegte Quotation-Umgebung (für Zitate)
\newcommand{\myboxquote}[1]{
	\begin{quotation}
		\colorbox{boxgray}{\parbox{0.78\textwidth}{#1}}
	\end{quotation}
	\vspace*{1mm}
}

% Definition von \vgl{#1}{#2}
\newcommand{\vgl}[2][]{vgl. \parencite[#1]{#2}}


\makeatletter
\zref@newprop*{appsec}{}
\zref@addprop{main}{appsec}

% Definition von \applabel{#1: label}{#2: text}
% von \appsec{#1: text}{#2: label} zur Erzeugung des Labels verwendet)
\def\applabel#1#2{%
	\zref@setcurrent{appsec}{#2}%
	\zref@wrapper@immediate{\zref@label{#1}}%
}

% Definition von \appref{#1: label}
% anstelle \ref{#1: label} zum referenzieren von Anhängen verwenden)
\def\appref#1{%
	\hyperref[#1]{\zref@extract{#1}{appsec}}%
}
\makeatother

% Definition von \appsection{#1: text}{#2: label}
% Ersetzt \section{#1: text} und \label{#2: label} für Anhänge)
\newcommand{\theappsection}[1]{Anhang \Alph{section}:~\protect #1}
\newcommand{\appsection}[2]{
	\addtocounter{section}{1}
	\phantomsection
	\addcontentsline{toc}{section}{\theappsection{#1}}
	\markboth{\theappsection{#1}}{}

	\section*{\theappsection{#1}}
	\applabel{#2}{Anhang \Alph{section}}
	\label{#2}
}

% Nummerierung der Formeln
\renewcommand{\theequation}{\arabic{equation}}

%%%%%%%%%%%%% Index, Abkürzungsverzeichnis und Glossar erstellen %%%%%%%%%%%%
\makeindex
\makenomenclature

% Festlegung der Art der Zitierung (Havardmethode: Abkuerzung Autor + Jahr) %
\usepackage[style=authoryear,language=ngerman]{biblatex}
\addbibresource{literatur.bib}


%%%%%%%%%%%%%%%%%%%%%%%%%%%%%%% PDF-Optionen %%%%%%%%%%%%%%%%%%%%%%%%%%%%%%%%
\hypersetup{
	bookmarksopen=false,
	bookmarksnumbered=true,
	bookmarksopenlevel=0,
	pdftitle=\myTopic,
	pdfsubject=\myTopic,
	pdfauthor=\myAutor,
	pdfborder=0,
	pdfstartview=Fit,
	pdfpagelayout=SinglePage
}

%%%%%%%%%%%%%%%%%%%%%%%%%%%% Kopf- und Fußzeile %%%%%%%%%%%%%%%%%%%%%%%%%%%%%
\pagestyle{fancy}
\fancyhf{}
\fancyhead[R]{\thepage}                         % Kopfzeile rechts bzw. außen
\renewcommand{\headrulewidth}{0.5pt}            % Kopfzeile rechts bzw. außen

%%%%%%%%%%%%%%%%%%%%%%%%% Layout und Beschriftungen %%%%%%%%%%%%%%%%%%%%%%%%%
\onehalfspacing                % Zeilenabstand: 1.5 (Standard: 1.2)
\setlist{noitemsep}            % Zeilenabstand für items auf 1 setzen

\addto\captionsngerman{        % Tabllen- und ildungsunterschriften ändern
  \renewcommand{\figurename}{Abb.}
  \renewcommand{\tablename}{Tab.}
}
\numberwithin{table}{section}                               % Tabellennummerierung je Sektion zurücksetzen
\numberwithin{figure}{section}                              % Abbildungsnummerierung je Sektion zurücksetzen
\renewcommand{\thetable}{\arabic{section}.\arabic{table}}   % Tabellennummerierung mit Section
\renewcommand{\thefigure}{\arabic{section}.\arabic{figure}} % Abbildungsnummerierung mit Section
\renewcommand{\thefootnote}{\arabic{footnote}}              % Sektionsbezeichnung von Fußnoten entfernen

\renewcommand{\multfootsep}{, }                             % Mehrere Fußnoten durch ", " trennen

%%%%%%%%%%%%%%%%%%%%%%%%%%%%%%% Listingstyle %%%%%%%%%%%%%%%%%%%%%%%%%%%%%%%%
\lstset{
	basicstyle=\ttfamily\scriptsize,
	commentstyle=\color{commentgray}\textit,
	showstringspaces=false,
	stringstyle=\color{darkgreen},
	keywordstyle=\color{blue},
	numbers=left,
	numberstyle=\tiny,
	stepnumber=1,
	numbersep=15pt,
	tabsize=2,
	fontadjust=true,
	frame=single,
	backgroundcolor=\color{boxgray},
	captionpos=b,
	linewidth=0.94\linewidth,
	xleftmargin=0.1\linewidth,
	breaklines=true,
	aboveskip=16pt
}

%%%%%%%%%%%%%%%%%%%%%%%%%%%%%%%%%%%%%%%%%%%%%%%%%%%%%%%%%%%%%%%%%%%%%%%%%%%%%
%%                                                                         %%
%% \/   \/      Bitte hier nur bei Bedarf Änderungen vornehmen     \/   \/ %%
%%                                                                         %%
%%%%%%%%%%%%%%%%%%%%%%%%%%%%%%%%%%%%%%%%%%%%%%%%%%%%%%%%%%%%%%%%%%%%%%%%%%%%%

%Seiten und Kapitel einbinden
\begin{document}
	\pagenumbering{Roman}
	% Das Titelblatt wird automatisch ausgewählt. Keine Änderung hier
	\ifcase\myType
            \begin{titlepage}
	\begin{center}
		\vspace*{1cm}
		\LARGE\bf\myTopic\\
		\Large\rm\mySubTopic\\
		\vspace*{2cm}
		\bf \myProjNumber.~Projektarbeit\\
		\vspace*{0.5cm}\singlespacing
		\normalsize\rm
		\ifcase\myNeedForFieldOfStudy
    		an der Fakultät für Wirtschaft\\
    		im Studiengang \myStudyProgram \\
            \or 
                an der Fakultät für Wirtschaft\\
    		im Studiengang \myStudyProgram \\
                in der Fachrichtung \myFieldOfStudy\\
            \else
            \fi
		\vspace*{0.5cm}\singlespacing
		an der\\
		DHBW Ravensburg
		\vfill
	\end{center}
	\begin{tabular}{lll}
		Autor/-in: &\myAutor\\
            Kurs: &\myCourse\\
            DHBW-Betreuer/-in: &\myProf\\
		Dualer Partner: &\myCompany\\
		Begleitperson Dualer Partner: &\myCompanion \\
		Abgabefrist: &\myEndDate
	\end{tabular}
	\newline
	\vspace*{1cm}
	\newline
	\begin{tabularx}{\textwidth}{l@{\extracolsep\fill}r}
	  % Unterschrift des verantwortlichen Ausbilders&\\
	 % (oder des Personalverantwortlichen)&\rule{6cm}{0.3mm}\\
	\end{tabularx}
\end{titlepage}
\newpage
\setcounter{page}{2}

        \or
            \begin{titlepage}
	\begin{center}
		\vspace*{2cm}
		\LARGE\bf\myTopic\\
		\Large\rm\mySubTopic\\
		\vspace*{3cm}
		\bf Bachelorarbeit\\
		\normalsize\rm
		\vspace*{0.5cm}\singlespacing
		für die\\
		Prüfung zum Bachelor of \myTypeOfBachelor\\
		\vspace*{0.5cm}\singlespacing
            \ifcase\myNeedForFieldOfStudy
    		an der Fakultät für Wirtschaft\\
    		im Studiengang \myStudyProgram \\
            \or 
                an der Fakultät für Wirtschaft\\
    		im Studiengang \myStudyProgram \\
                in der Fachrichtung \myFieldOfStudy\\
            \else
            \fi
		\vspace*{0.5cm}\singlespacing
		an der\\
		DHBW Ravensburg
		\vfill
	\end{center}
	\begin{tabular}{ll}
		Autor/-in: &\myAutor\\
            Kurs: &\myCourse\\
            DHBW-Betreuer/-in: &\myProf\\
		Dualer Partner: &\myCompany\\
		Begleitperson Dualer Partner: &\myCompanion \\
		Abgabefrist: &\myEndDate
	\end{tabular}
\end{titlepage}
\newpage
\setcounter{page}{2}

        \or
            \begin{titlepage}
	\begin{center}
		\vspace*{1cm}
		\LARGE\bf\myTopic\\
		\Large\rm\mySubTopic\\
		\vspace*{2cm}
		\bf \myTypeOfWork \\
		\vspace*{0.5cm}\singlespacing
		\normalsize\rm
		\ifcase\myNeedForFieldOfStudy
    		an der Fakultät für Wirtschaft\\
    		im Studiengang \myStudyProgram \\
            \or 
                an der Fakultät für Wirtschaft\\
    		im Studiengang \myStudyProgram \\
                in der Fachrichtung \myFieldOfStudy\\
            \else
            \fi
		\vspace*{0.5cm}\singlespacing
		an der\\
		DHBW Ravensburg
		\vfill
	\end{center}
	\begin{tabular}{ll}
		\ifcase\myNames
    		Autor/-in:
            \or 
                Autoren/Autorinnen:
            \else
            \fi
            &\myAutor\\
            Modul: &\myModule\\
		Dozent/-in:&\myProf\\
		Abgabefrist:&\myEndDate
	\end{tabular}
	
\end{titlepage}
\newpage
\setcounter{page}{2}

        \else
        \fi

	\pagestyle{fancy}
	\ifcase\myBlockingNote

\or
    \begin{titlepage}
	\begin{center}
		\vspace*{1cm}
		\Huge\bf Sperrvermerk\\
		\vspace*{2cm}
		\large\rm
		
		\begin{quotation}
			\hspace*{-1.75em}
			\parbox{0.85\textwidth}{\singlespacing Der Inhalt dieser Arbeit darf weder als Ganzes noch in Auszügen Personen außerhalb des 
Prüfungs- und Evaluationsverfahrens zugänglich gemacht werden, sofern keine 
anderslautende Genehmigung des Dualen Partners vorliegt.}
		\end{quotation}
		\vspace*{0.5cm}
		\begin{quotation}
			\hspace*{-1.5em}
			\parbox{\textwidth}{
				\singlespacing
				\begin{tabularx}{0.83\textwidth}{l@{\extracolsep\fill}l}
					\rule{4cm}{0.3mm}&\rule{4cm}{0.3mm}\\
					\large
					Ort, Datum&\large Unterschrift
				\end{tabularx}}
			\end{quotation}
		\end{center}
	\end{titlepage}
	\newpage
	\setcounter{page}{3}

\else
\fi
\tableofcontents
\newpage


\ifcase\myAbrevationDirectory

\or
    \section*{Abkürzungsverzeichnis}
\addcontentsline{toc}{section}{Abkürzungsverzeichnis}

\begin{acronym} [DHBW]
    \acro{ESS}{Enterprise Social Software}
\end{acronym}

\printacronym
\newpage
\else
\fi
\ifcase\myIllustrationDirectory

\or
    \listoffigures
\newpage

\else
\fi
\ifcase\myTableDirectory

\or
    \listoftables
\newpage

\else
\fi
\ifcase\myListingDirectory

\or
    %Listingnummering je Sektion zurücksetzen
\numberwithin{lstlisting}{section}
%Listingnummerierung mit Section
\renewcommand{\thelstlisting}{\arabic{section}.\arabic{lstlisting}}
%Listingsverzeichnis in das Inhaltsverzeichnis aufnehmen.
\renewcommand{\lstlistingname}{Listingsverzeichnis}
\renewcommand{\lstlistoflistings}{\begingroup
\tocchapter
\tocfile{\lstlistingname}{lol}
\endgroup}

\lstlistoflistings
\newpage

\renewcommand{\lstlistingname}{Listing}

\else
\fi
\ifcase\myFormularDirectory

\or
    \section*{Formelverzeichnis}
\renewcommand{\theequation}{FV.\arabic{equation}} % Nummerierung anpassen
\counterwithin{equation}{section}

% Definition der Liste mit Punkten ohne Klammern
\newcommand{\dotfillref}[2]{#1 \dotfill #2}

\begin{enumerate}[label={}, leftmargin=0pt, itemindent=*]
    \item \dotfillref{\(E = mc^2\)}{\ref{eq:einstein}}
\end{enumerate}
\else
\fi
\ifcase\mySymbolDirectory

\or
    \addcontentsline{toc}{section}{Symbolverzeichnis} % Falls im Inhaltsverzeichnis gewünscht
\renewcommand{\nomname}{Symbolverzeichnis}

\nomenclature{$E$}{Energie (Joule)}
\nomenclature{$m$}{Masse (Kilogramm)}
\nomenclature{$c$}{Lichtgeschwindigkeit im Vakuum (m/s)}
\printnomenclature

\else
\fi


	% Kapitel
	\pagestyle{fancy}
	\fancyhead[L]{\nouppercase{\leftmark}}                               % Kopfzeile links bzw. innen
	\pagenumbering{arabic}

	%%%%%%%%%%%%%%%%%%%%%%%%%%%%%%%%%%%%%%%%%%%%%%%%%%%%%%%%%%%%%%%%%%%%%%%%%%%%%
%%                                                                         %%
%% \/   \/      Erklärung der Begriffe						       \/   \/ %%
%%                                                                         %%
%%%%%%%%%%%%%%%%%%%%%%%%%%%%%%%%%%%%%%%%%%%%%%%%%%%%%%%%%%%%%%%%%%%%%%%%%%%%%
%% Überschriften und Gliederung				%
\section{Einleitung}
\label{sec:einleitung}
% Einleitung der Arbeit

\section{Zusammenarbeitssysteme}
\label{sec:zusammenarbeitssysteme}
\subsection{Integrierte Bürosofware}
\label{subsec:integrierte_bürosoftware}
\subsection{Workflow-Management-Systeme}
\label{subsec:workflow_management_systeme}
\subsection{Dokumentenmanagementsysteme}
\label{subsec:dokumentenmanagementsysteme}
\subsection{Content-Management-Systeme}
\label{subsec:content_management_systeme}
\subsection{Workgroup-Computing}
\label{subsec:workgroup_computing}
\subsection{Enterprise 2.0}
\label{subsec:enterprise_2_0}
\section{Fazit}
Ich bin ein Fazit.

	% Anhang
	\renewcommand{\thetable}{\Alph{section}.\arabic{table}}              % Tabellennummerierung mit Section
	\renewcommand{\thefigure}{\Alph{section}.\arabic{figure}}            % Abbildungsnummerierung mit Section
	\renewcommand{\thelstlisting}{\Alph{section}.\arabic{lstlisting}}    % Listingsnummerierung mit Section

	% Abschluss
	\printbibliography
\newpage

        \include{pages/22_KiVerzeichnis}

	\begin{appendix}
		%% Include für alle Anhänge

%% Coverpage des Anhangs oder Einzelseite (1)
%% Mit Anhangstitel


	\end{appendix}

        \ifcase\myDecOfIndependence

        \or
            \thispagestyle{empty}

\begin{comment}
    \addcontentsline{toc}{section}{Selbständigkeitserklärung}
\end{comment}

\begin{center}
	\vspace*{2cm}
	\Huge\bf Selbständigkeitserklärung\\
	\vspace*{3cm}
	\large\rm\singlespacing 
	Ich versichere hiermit, dass ich meine\\ \ifcase\myType Seminararbeit \or Projektarbeit \or Bachelorarbeit\else\fi ~mit dem Thema\\
	\vspace*{2cm}
	\Large\bf\myTopic\\
	\Large\rm\mySubTopic\\
	\vspace*{2cm}
	\large\rm
	\singlespacing 
	selbständig verfasst und keine anderen als die angegebenen\\Quellen und Hilfsmittel benutzt habe. Ich versichere  zudem, 
	dass die eingereichte elektronische  Fassung 
	mit der gedruckten Fassung übereinstimmt.\\
	\vfill
	\begin{tabularx}{\textwidth}{l@{\extracolsep\fill}r}
            \myPlace, \myKiDate & \\
            \rule{7cm}{0.3mm} & \rule{7.55cm}{0.3mm} \\
        \end{tabularx}
        \begin{tabularx}{\textwidth}{*{2}{>{\arraybackslash}X}}
            Ort, Datum & Unterschrift \\
        \end{tabularx}
\end{center}

        \else
        \fi

        
\end{document}

%%%%%%%%%%%%%%%%%%%%%%%%%%%%%%%%%%%%%%%%%%%%%%%%%%%%%%%%%%%%%%%%%%%%%%%%%%%%%
%%                                                                         %%
%% /\   /\         Ab hier keine Änderungen mehr vornehmen         /\   /\ %%
%%                                                                         %%
%%%%%%%%%%%%%%%%%%%%%%%%%%%%%%%%%%%%%%%%%%%%%%%%%%%%%%%%%%%%%%%%%%%%%%%%%%%%%
